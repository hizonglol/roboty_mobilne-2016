\documentclass[a4paper]{article}
\usepackage{polski}
\usepackage[utf8]{inputenc}
\usepackage{enumerate}
\usepackage{hyperref}

\title{Roboty mobilne - projekt}
\date{}

\begin{document}
\maketitle

\begin{enumerate}

\item Skład grupy:

Dominik Hebda 209168 \\
Filip Malinowski 209193

\item Opis projektu:

Wykonamy egzoszkielet dłoni sterowany siłowo poprzez odczyty z czujników nacisku umieszczonych wewnątrz egzoszkieletu.

\item Założenia projektowe:

Egzoszkielet powinien wspomagać ruchy dłoni, uruchamiając się dopiero po napotkaniu większego oporu przez dłoń. Oznacza to tyle, że jeśli palce naciskając na wnętrze obudowy wywrą siłę większą od pewnego progu to egzoszkielet uruchomi napędy.
Do zrealizowania odczytu nacisku palców wnętrze rękawicy będzie wyposażone w czujniki nacisku zmieniające pod wpływem nacisku swoją rezystancję, badające w ten sposób nacisk odpowiednich części dłoni na wnętrze egzoszkieletu.
Sterowanie będzie zrealizowane za pomocą pętli histerezy połączonej z regulatorem PID. Gdy siła nacisku dojdzie do progu zadanego w programie sterowania, uruchomi się regulator, którego zadaniem będzie utrzymywanie siły nacisku palców na wnętrze obudowy na poziomie będącym przeskalowaniem w dół rzeczywistej siły nacisku potrzebnej do wykonania ruchu.

W pracy energooszczędnej egzoszkielet powinien być wyłączony, a napędy przegubów powinny działać jako prądnice, które w trakcie normalnego ruchu człowieka powinny ładować akumulatory.

\item Plan pracy:

W planie pracy naszego projektu przewidujemy trzy kamienie milowe:
\begin{enumerate}[I -]
\item przeprowadzenie badań na temat materiałów, napędów, zasilania, sensorów, warstwy sprzętowej realizującej pomiary i algorytm oraz schematu blokowego algorytmu sterowania odpowiadającym naszym założeniom projektowym
\item opis opracowanej przez nas konstrukcji egzoszkieletu dłoni
\item symulacja ręki realizująca elementy opisu egzoszkieletu
\end{enumerate}

\newpage

\item Podział prac:

Podział prac zdefiniowaliśmy do drugiego kamienia milowego. Trzeci sprecyzujemy po wykonaniu pierwszego.
\begin{enumerate}[I]
\item kamień milowy:
\begin{itemize}
\item Dominik Hebda: \\
materiały, zasilania, sensory
\item Filip Malinowski \\
napędy, warstwa sprzętowa, schemat blokowy algorytmu
\end{itemize}
\item kamień milowy:
\begin{itemize}
\item Dominik Hebda: \\
model komputerowy dłoni
\item Filip Malinowski \\
model komputerowy dłoni oraz algorytm sterowania
\end{itemize}
\end{enumerate}

\item Kosztorys:

Nie przewidujemy kosztów związanych z wykonywaniem projektu.

\item Doręczenie:
\begin{enumerate}[I]
\item kamień milowy - 12. kwietnia 2016
\item kamień milowy - 7. czerwca 2016
\end{enumerate}

\item Zarządzanie projektem:
\begin{itemize}
\item Git - do przechowywania i rozwijania dokumentacji oraz oprogramowania związanego z projektem.
\href{https://github.com/hizonglol/roboty_mobilne-2016}{Adres internetowy repozytorium.}
\item LaTeX - do tworzenia dokumentacji projektu
\end{itemize}
\end{enumerate}

\end{document}